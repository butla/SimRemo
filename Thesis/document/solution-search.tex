\chapter{Poszukiwanie rozwiązania}
Jak w ogóle dojść do tego  co i jak zrobić, żeby dało nam to, co mamy w celach? Jak się odnaleźć w tym wszystkim?
Wyjaśnić jak wytaczałem sobie drogę działania i jak metodyki dobierałem

Jak teoretycznie możnaby zrobić web service'y na Androidzie? Android jest tutaj przykładem ograniczonego systemu.
W sumie jakiś powód musięli mieć, żeby ich nie zamieszać. Jakie jeszcze dodatkowe usprawnienia byłyby do zrobienia (tłumaczenie klas)?

Najpierw zobaczmy, co jest.

Napisać, co jest potrzebne dla remotingu: jakaś tożsamość, adres obiektu, oznaczenia metod, argumentów.

Wprowadzenie nowego standardu nie jest dobrą rzeczą, chyba, że ma się ogromną rzeszę zwolenników lub poważne przesłanki. Kto wie, może takie mam. Ale na ogół lepiej kombinować istniejące rozwiązania tak, żeby zewnętrzne komponenty miały szansę na współpracę z nimi.

A czemu nie MONO/Xamarin?
\section{Pokrewne rozwiązania}
\label{existing_frameworks}
Tu o bibliotekach, frameworkach robiących to co chcemy albo rzeczy podobne.

Opisać jedną wspólną procedurę oceny dla każdej biblioteki: znalezienie w necie, instalacja, stworzenie prostej aplikacji, stworzenie bardziej skomplikowanej aplikacji (jakaś wymieniająca kilka ustalonych skomplikowanych drzew obiektów) łączenie z .NETem?

Konkretne porównania osiągów w rozdziale o eksperymentach (chociaż tutaj już zrobić porównanie wysiłku programisty, po prostu zamieścić je na końcu). Tutaj napisać o funkcjonalności żeby w ogóle wiedzieć, czy to czegoś warte i na ile mam się inspirować. Poza tym, może się okazać że jest już coś, co robi to, ca ja bym chciał zrobić.

Te wszystkie do serializacji i soapów rzeczy to są, tutaj nic nie ma. Wszystko co by się chciało to jest, ale na dużą Javę, albo na pythona (ciężko przeportować, nie wiadomo, czy da się na pewno) Apache Thrift, Google Protocol Buffers, ZeroMQ (http://stackoverflow.com/questions/8062212/difference-between-apache-thrift-and-zeromq)

Przy każdej technologii pokazać implementacje w WCF i Javie normalnej (bo ona będzie takim punktem odniesienia) i zrobić porównania prędkości, bezpieczeństwa itp.

Każdy framework opisać ile plików, jaką objętość trzeba ściągnąć. Ile kroków instalacji, ile rzeczy trzeba zrobić, żeby wystawić jakiś helloWorld serwis i jakiś serwis z własnymi danymi. Zrobić zestawienie frameworków (tabelę), czy pozwalają na wystawianie serwisów, czy tylko korzystanie. Czy pozwalają na polimorficzne argumenty.

GSOAP

jakieś może znaleźć frameworki pozwalające oznakować klasy do serializacji (binding) do xmla albo jsona

Python EVE (Rest framework)
\url{http://www.blog.pythonlibrary.org/2014/08/13/jsonpickle-turning-python-pickles-into-json/}

\url{http://www.webopedia.com/TERM/W/Web_Services.html}\\
\url{http://thrift.apache.org/static/files/thrift-20070401.pdf}\\
reliable sessions (w sumie też by mi się przydał jakiś mechanizm, który umożliwi komunikację w niestabilnym środowisku Internetu)\\
\url{http://blogs.msdn.com/b/shycohen/archive/2006/02/20/535717.aspx}\\
\url{http://docs.xamarin.com/guides/cross-platform/application_fundamentals/web_services}\\
\url{http://wsme.readthedocs.org/en/latest/}\\

Apache Thrift, Google Protocol Buffers, ZeroMQ (\url{http://stackoverflow.com/questions/8062212/difference-between-apache-thrift-and-zeromq})\\
\url{http://thrift.apache.org/static/files/thrift-20070401.pdf}\\
\url{http://blogs.msdn.com/b/shycohen/archive/2006/02/20/535717.aspx}\\
Xamarin\\
\url{http://docs.xamarin.com/guides/cross-platform/application_fundamentals/web_services/}\\

JiBX!!! Tworzenie klas ze schemy i robienie wsdl z javy\\
Wsdl2Java: tworzenie klas z wsdl, głównie chodzi o klasy danych\\

Serializacja:\\
\url{http://simple.sourceforge.net/}\\
\url{http://code.google.com/p/dbdroid-remoting/} - klient web serviców serializujący i deserializujący xmle\\
Generacja javovego kodu z bindingu:\\
XSD2Java – generuje kod java z xsd. Mocno niedorobione.\\
Inne:\\
\url{http://stackoverflow.com/questions/6920175/how-to-generate-java-classes-from-wsdl-file}\\
\url{http://blog.tourgeek.com/2011/12/xml-data-binding-for-java-on-android.html}\\
PODOBNE, ALE NIE PRZYDATNE:\\

\url{http://forum.springsource.org/showthread.php?129058-Spring-Remoting-for-Android}\\
\url{http://www.themidnightcoders.com/fileadmin/docs/java/v4/index.html?android}\\
\url{http://developer.android.com/guide/components/aidl.html} - AIDL, androidowa komunikacja między procesami\\


\subsection{WCF}
\url{http://msdn.microsoft.com/en-us/library/system.runtime.serialization.json.datacontractjsonserializer.aspx}\\
Generowanie z JAX-WS nie do końca działało.

Opisać o co chodzi z tym polimorfizmem, że w WCF trzeba wypisywać KnownType, co jest robieniem wzajemnych zależności, co jest straszną praktyką programowania (źródło).

\subsection{JAX-WS}
Niestandardowe podejście z użyciem JSONa zamiast XMLa\url{http://jax-ws-commons.java.net/json/}\\

\subsection{Jackson}
\url{http://www.cowtowncoder.com/blog/archives/2010/03/entry_372.html}\\
\url{http://stackoverflow.com/questions/8368873/deserialize-json-string-generated-from-net-using-jackson}
\url{http://stackoverflow.com/questions/10329706/json-deserialization-into-another-class-hierarchy-using-jackson}
\url{http://stackoverflow.com/questions/14454028/polymorphic-serialization-of-collections-with-custom-serializer-in-jackson}
\url{http://stackoverflow.com/questions/12350571/how-can-i-change-global-type-information-format-in-jackson}

sprawdzić jak się wymienia ze wszystkim opisanym adnotacjami (dodać jeszcze rozszerzenie defaulttyperesolvera), wtedy sprawdzić mixed iny; lista objectów, na którą wpakuję stringi i inty

\subsection{Restlet}
\url{http://restlet.org/learn/tutorial/2.1/#/docs_2.0/13-restlet/275-restlet/266-restlet.html}\\
\url{http://wiki.fasterxml.com/JacksonHowToIgnoreUnknown}\\

\subsection{Spring}
\subsection{Crest}
\subsection{Ksoap}

\subsection{I-jetty}
Jest to serwer, może ma jakiś kontener Web-serviceów?

\subsection{Porównanie możliwości tych technologii}

Jakieś niestandardowe (third-party) narzędzia tworzące WSDLa z Javy i .NETa i z WSDLa klasy? To samo do schemy i ze schemy?

\section{Moje pomysły na rozwiązanie}
Wymienić, zrobić POCe, porównać, wybrać i uzasadnić wybór. Wypracować jaką metodę odcięcia pracy na POCem? Że jak będę widział, że da się coś zrobić, ale jest to za trudne to kończyć pracę.

Namespace'y są ważną rzeczą w SOAPie. Ja twierdzę, że nie są potrzebne w tak rozbudowanej formie. Może być jedno standardowe nazewnictwo z kropkami. Zakładam taki jakby jeden globalny namespace na oba runtime'y (klient i serwer). Fakt, że nawet jedna strona może mieć kilka przestrzeni nazw (parę class loaderów albo assembly cache'y) ale nie chcę tego robić z uwagi n utrudnienia w integracji (tu można napisać jakby to mogło wyglądać, ale nie trzeba, mogę po prostu to olać).

Napisanie wszystkiego w C, Javie albo nawet Pythonie. Z czym by się to wiązało. Chcemy żeby było łatwo, co by trzeba było zrobić wtedy?
Podłączanie się pod istniejące rozwiązania Axis w C, portowanie Pythona, tłumaczenie JibXem

Zdecydowałem się to zrobić web servicami opartymi o SOAP, blabla, jednolite dla javy tej i normalnej
Tu już konkretnie jakie biblioteki, jak konkretniej to będzie wyglądać

Tyson? Typed JSON?

a)	Rozpoznanie istniejących rozwiązań dotyczących wiązania danych (ang. data binding) z XML oraz wystawiania usług sieciowych (ang. web services) w językach Java oraz C++.
Zaprojektowanie i wykonanie biblioteki pozwalającej na przyjmowanie, przetwarzanie, tworzenie i wysyłanie wiadomości SOAP, generowanie plików WSDL na bazie kodu oraz na wiązanie danych z języka C\# do języka Java z zakresem bibliotek dostępnym na Androidzie. Akceptowalne jest również stworzenie biblioteki, w której komunikacja oparta zostałaby o protokół TCP/IP i serializację obiektów do XML tak, by były rozumiane zarówno przez kod w języku Java, jak i w języku C\#.

Użyć w Pythonie Flaska do tłumaczenia JSONa? Albo jakiegoś innego tłumacza python-JSON.

Może zamiast tylu podsekcji najpierw wyliczenie, krótki opis i decyzja, czy należy się tym zajmować? Dopiero te, które wydawałyby się bardziej interesujące dostają swoją własną podsekcję.

\subsection{własna metoda oparta o JSON z oznaczaniem typów}
Póki co wiem, że będzie JSON z opisanymi typami. Kanał komunikacyjny powinien być dowolny. 

Może zamiast równoważnych implementacji na pythona, C\# i javę zrobić tylko pythona i wrappery w jythonie i iron pythonie?

\subsection{serwer web serviców z narzędziami do generacji kodu itp}
Generator wsdl na Androida + tworzenie bindingów do xsd + Parsowanie SOAPów

JiBX!!! Tworzenie klas ze schemy i robienie wsdl z javy
Wsdl2Java: tworzenie klas z wsdl, głównie chodzi o klasy danych

Weryfikacja: tworzę graf w xmlu. Są obiekty, które mogą mieć dzieci (tablice, listy). Próbuję wygenerować C\# dla porównania. Zapisuję grafy, które nie działają (w trakcie automatycznych testów). Zrobić wizualizator grafów.

Co by trzeba było po kolei zrobić w normalnej javie? (tłumaczenia do xsd, ujednolicanie namespaców, mamy do czynienia z jakimiś ArrayOfString, dziedziczenie chyba słabo działa przy generacji, nie da się używać DataContractSerializera żeby nie serializować części klas ogólnie syf, ale wszystko udokumentować). 

\subsection{Własny serwer REST}
Tu dyskusja o implementacji serwera RESTowego na Androidzie \url{https://groups.google.com/forum/?fromgroups=#!topic/android-developers/vgkXg1P8iBg}

\subsection{remoting przez serializację}
Po prostu serializujemy tak samo/

\subsection{portowanie rozwiązań w C na Androida?}
Axis. Może częściowe używanie go spod javy. W sumie można sportować też jakieś inne rzeczy. Np.\ pythonowe web service'y (pyro, twisted).

Sprawdzić Python Remote Objects (Pyro) na pythonie 2, 3 (też na warstwie skryptowej Androida). Porównać z Twistedem, napisać o próbach portowania wymaganych bibliotek w C na Androida. \url{http://pythonhosted.org/Pyro4/intro.html#performance}

sl4a, py4a, kompilacja na ubuntu z użyciem NDK i Cythona
distutils.core zamienić na setuptools w setup.py

\subsection{Zcentralizowane pseudo web servicy (albo nawet i nie)}
Też potrzebny serializator / deserializator + bindingi. To był ten pomysł, że to komp pinga serwer (Androida), żeby ten sobie pobrał rozkaz i to Android woła web service na kompie. Trzeba by było używać Ksoap albo innego ścierwa. Zobaczyć, czy to w ogóle miałoby sens. Czy tymi Ksoapami da się coś sensownie zrobić.

\subsection{Wybrane rozwiązanie}
Jakie i dlaczego?

Dlaczego nie Soap, bindingi itp.?
