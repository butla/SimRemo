\chapter{Pokrewne technologie}
Tu o bibliotekach, frameworkach robiących to co chcemy albo rzeczy podobne.

Walnąć tabelę z obecnością każdej z poszukiwanych cech, może też z czasem isntalacji

Opisać jedną wspólną procedurę oceny dla każdej biblioteki: znalezienie w necie, instalacja, stworzenie prostej aplikacji, stworzenie bardziej skomplikowanej aplikacji (jakaś wymieniająca kilka ustalonych skomplikowanych drzew obiektów) łączenie z .NETem?

Konkretne porównania osiągów w rozdziale o eksperymentach (chociaż tutaj już zrobić porównanie wysiłku programisty, po prostu zamieścić je na końcu). Tutaj napisać o funkcjonalności żeby w ogóle wiedzieć, czy to czegoś warte i na ile mam się inspirować. Poza tym, może się okazać że jest już coś, co robi to, ca ja bym chciał zrobić.

Te wszystkie do serializacji i soapów rzeczy to są, tutaj nic nie ma. Wszystko co by się chciało to jest, ale na dużą Javę, albo na pythona (ciężko przeportować, nie wiadomo, czy da się na pewno) Apache Thrift, Google Protocol Buffers, ZeroMQ (http://stackoverflow.com/questions/8062212/difference-between-apache-thrift-and-zeromq)

Przy każdej technologii pokazać implementacje w WCF i Javie normalnej (bo ona będzie takim punktem odniesienia) i zrobić porównania prędkości, bezpieczeństwa itp.

Każdy framework opisać ile plików, jaką objętość trzeba ściągnąć. Ile kroków instalacji, ile rzeczy trzeba zrobić, żeby wystawić jakiś helloWorld serwis i jakiś serwis z własnymi danymi. Zrobić zestawienie frameworków (tabelę), czy pozwalają na wystawianie serwisów, czy tylko korzystanie. Czy pozwalają na polimorficzne argumenty.

GSOAP

jakieś może znaleźć frameworki pozwalające oznakować klasy do serializacji (binding) do xmla albo jsona

Python EVE (Rest framework)
\url{http://www.blog.pythonlibrary.org/2014/08/13/jsonpickle-turning-python-pickles-into-json/}

\url{http://www.webopedia.com/TERM/W/Web_Services.html}\\
\url{http://thrift.apache.org/static/files/thrift-20070401.pdf}\\
reliable sessions (w sumie też by mi się przydał jakiś mechanizm, który umożliwi komunikację w niestabilnym środowisku Internetu)\\
\url{http://blogs.msdn.com/b/shycohen/archive/2006/02/20/535717.aspx}\\
\url{http://wsme.readthedocs.org/en/latest/}\\

Apache Thrift, Google Protocol Buffers, ZeroMQ (\url{http://stackoverflow.com/questions/8062212/difference-between-apache-thrift-and-zeromq})\\
\url{http://thrift.apache.org/static/files/thrift-20070401.pdf}\\
\url{http://blogs.msdn.com/b/shycohen/archive/2006/02/20/535717.aspx}\\

Xamarin\\
\url{http://docs.xamarin.com/guides/cross-platform/application_fundamentals/web_services/}\\
Ddawniej mono dla androida, nie można robić web serviceów (czyli nie ma całego .NETa). Z api Androida nie wiem, jak jest.

JiBX!!! Tworzenie klas ze schemy i robienie wsdl z javy\\
Wsdl2Java: tworzenie klas z wsdl, głównie chodzi o klasy danych\\

Serializacja:\\
\url{http://simple.sourceforge.net/}\\
\url{http://code.google.com/p/dbdroid-remoting/} - klient web serviców serializujący i deserializujący xmle\\
Generacja javovego kodu z bindingu:\\
XSD2Java – generuje kod java z xsd. Mocno niedorobione.\\
Inne:\\
\url{http://stackoverflow.com/questions/6920175/how-to-generate-java-classes-from-wsdl-file}\\
\url{http://blog.tourgeek.com/2011/12/xml-data-binding-for-java-on-android.html}\\
PODOBNE, ALE NIE PRZYDATNE:\\

\url{http://forum.springsource.org/showthread.php?129058-Spring-Remoting-for-Android}\\
\url{http://www.themidnightcoders.com/fileadmin/docs/java/v4/index.html?android}\\
\url{http://developer.android.com/guide/components/aidl.html} - AIDL, androidowa komunikacja między procesami\\


\section{WCF}
\url{http://msdn.microsoft.com/en-us/library/system.runtime.serialization.json.datacontractjsonserializer.aspx}\\
Generowanie z JAX-WS nie do końca działało.

Opisać o co chodzi z tym polimorfizmem, że w WCF trzeba wypisywać KnownType, co jest robieniem wzajemnych zależności, co jest straszną praktyką programowania (źródło).

\section{JAX-WS}
Niestandardowe podejście z użyciem JSONa zamiast XMLa\url{http://jax-ws-commons.java.net/json/}\\

\section{Jackson}
\url{http://www.cowtowncoder.com/blog/archives/2010/03/entry_372.html}\\
\url{http://stackoverflow.com/questions/8368873/deserialize-json-string-generated-from-net-using-jackson}
\url{http://stackoverflow.com/questions/10329706/json-deserialization-into-another-class-hierarchy-using-jackson}
\url{http://stackoverflow.com/questions/14454028/polymorphic-serialization-of-collections-with-custom-serializer-in-jackson}
\url{http://stackoverflow.com/questions/12350571/how-can-i-change-global-type-information-format-in-jackson}

sprawdzić jak się wymienia ze wszystkim opisanym adnotacjami (dodać jeszcze rozszerzenie defaulttyperesolvera), wtedy sprawdzić mixed iny; lista objectów, na którą wpakuję stringi i inty

\section{Restlet}
\url{http://restlet.org/learn/tutorial/2.1/#/docs_2.0/13-restlet/275-restlet/266-restlet.html}\\
\url{http://wiki.fasterxml.com/JacksonHowToIgnoreUnknown}\\

\section{Spring}
\section{Crest}
\section{Ksoap}

\section{I-jetty}
Jest to serwer, może ma jakiś kontener Web-serviceów?

\section{Porównanie możliwości tych technologii}

Jakieś niestandardowe (third-party) narzędzia tworzące WSDLa z Javy i .NETa i z WSDLa klasy? To samo do schemy i ze schemy?
