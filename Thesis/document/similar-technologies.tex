\chapter{Pokrewne technologie (TODO)}
Tu o rozwiązaniach robiących to co chcemy albo rzeczy podobne. Przy użyciu każdego z tych frameworków spróbuję zrealizować mój cel, a więc dynamiczne web service'y na Androidzie. Czyli takie, jakie można robić przy pomocy WCF.

Może nawet okaże się, że jest tutaj coś, co spełnia moje oczekiwania? Albo wymaga tylko pewnego opakowania, żeby robić to, co ja chce?
Jeśli nie, to może chociaż mogę użyć jako moduł w swojej bibliotecę?

Dla każdego obiecującego frameworku powinienem wykonać i opisać test, jaki wykonałem, żeby sprawdzić czy nadaje się dla moich celów.

%Przy każdej technologii pokazać implementacje w WCF i Javie normalnej (bo ona będzie takim punktem odniesienia) i zrobić porównania prędkości, bezpieczeństwa itp.

%jakieś może znaleźć frameworki pozwalające oznakować klasy do serializacji (binding) do xmla albo jsona

%Konkretne porównania osiągów w rozdziale o eksperymentach (chociaż tutaj już zrobić porównanie wysiłku programisty, po prostu zamieścić je na końcu). Tutaj napisać o funkcjonalności żeby w ogóle wiedzieć, czy to czegoś warte i na ile mam się inspirować.
\section{Metoda analizy technologii}
Czyli z jakiej strony będę patrzył na frameworki, co będe sprawdzał, na co zwracał uwagę. Jak będzie jeden wzorzec to łatwiej zrobić tabelkę, czy coś.

W sumie będę sprawdzał, czy każda z technologii ma cechy, które opisałem w \ref{lib_requirements}.

\subsubsection{Użyteczność}
\begin{itemize}
	\item Jak przebiega instalacja? Jest oczywista, automatyczna? Wszystko łatwo znaleźć?
	\item Jak można uruchomić helloWorld?
	\item Na ile to elastyczne i potężne? Czyli jak łatwo można zrobić swoją robotę i to tak, jak się chce?
\end{itemize}

\subsubsection{Czy może być hostem}
Czyli czy pozwala na wystawianie serwisu, czy może tylko być klientem. Czy może ani tym, ani tym? Tylko w czymś możee pomóc?

\subsubsection{Sprawdzanie polimorfizmu}
Jak sprawdzam polimorfizm?
Przede wszystkim trzeba sprawdzić, czy metody spełniają założenia polimorfizmu. Poza tym templateowym, bo tego nie chce. Czyli jak z argumentami dziedziczącymi, jak z przeładowaniem metod. Jak z listami różnych elementów, które dziedziczą po typie bazowym.
Potem przyjrzenie się liście rzeczy, które stworzyłem na bazie doświadczenia z ktorymi silniki serializacyjne maja problemy, a sa ta sensowne przypadki. Np. listy list, listy obiektów o różnych typach (też listach). Puste tablice.

Przypadki testowe:
Co? Dlaczego?
Test podstawowy, którego będą wariacje:
Metoda taka i taka. Jeden parametr. To samo zwraca, to samo przyjmuje.

%algorithmic do kodu?

Kod podstawowej klasy danych. Jej obiekty będę przekazywane i~zwracane w~większości testów.
Szczegóły składni nie będą ważne, można traktować to jako pseudokod, ale faktycznie na ogół będzie to C\#.

\begin{lstlisting}[frame=single, caption={Przykładowa klasa danych}, label=kod:testData]
public class DaneA
{
	public int liczba;
	public string tekst;
}
\end{lstlisting}

Kod podstawowej metody testowej. Będzie ona wykonywana w~większości testów.
Po prostu zwraca lekko zmodyfikowaną kopię parametru.

\begin{lstlisting}[frame=single, caption={Przykładowa zdalna metoda}, label=kod:testMethod]
public DaneA MetodaA(DaneA dane)
{
	DaneA noweDane = new DaneA();
	noweDane.liczba = dane.liczba + 1;
	noweDane.tekst = dane.tekst + "xxx";	
	return noweDane;
}
\end{lstlisting}

\begin{tabular}{ | l | l | }
  \hline
	\textbf{Test} & \textbf{Uzasadnienie} \\
	\hline \hline
  4 & 5 \\
	\hline
  7 & 8 \\
	\hline
\end{tabular}

\subsubsection{Swoboda rozszerzania kodu}
Czy na tym, co się stworzy danym narzędziem można dalej budować. Zgodnie z obiektową zasadą reuse?

\section{WCF}
Potężny framework Web Serviceów od Microsoftu. Może posłużyć za wzór jak wszystko może działać. Może dość dynamicznie tworzyć hierarchie obiektów, które mogą funkcjonować jako usługi.
%\url{http://msdn.microsoft.com/en-us/library/system.runtime.serialization.json.datacontractjsonserializer.aspx}\\
%Generowanie z JAX-WS nie do końca działało.

%Opisać o co chodzi z tym polimorfizmem, że w WCF trzeba wypisywać KnownType, co jest robieniem wzajemnych zależności, co jest straszną praktyką programowania (źródło).

\section{JAX-WS}
Standardowy framework Javy do Web Serviceów. Podobne możliwości co WCF. Ja też niby chcę web service'y w Javie, ale na Androidzie, któremu JAX-WS został wycięty.
%Niestandardowe podejście z użyciem JSONa zamiast XMLa\url{http://jax-ws-commons.java.net/json/}\\

\section{gSOAP}
Pozwala tworzyć i konsumować Web Servicey z poziomu niezależnej biblioteki C++. Może będzie działać na Androidzie?
%http://trac.e-technik.uni-rostock.de/projects/ws4d-gsoap/wiki/AndroidNDK

\section{Xamarin}
Pozwala tworzyć w .NETcie aplikacje na Androida. Nie wspiera całości .NET.

Testowałem sobie na emulatorze, opis dopiero w dziale z implementacją, bo do całości pracy to się przydało.

%\url{http://docs.xamarin.com/guides/cross-platform/application_fundamentals/web_services/}\\
%Ddawniej mono dla androida, nie można robić web serviceów (czyli nie ma całego .NETa). Z api Androida nie wiem, jak jest.

\section{Jackson}
Konfigurowalna i dopracowana biblioteki do serializacji do JSONa. To może się przydać przy komponowaniu wywołań RPC.

%\url{http://www.cowtowncoder.com/blog/archives/2010/03/entry_372.html}\\
%\url{http://stackoverflow.com/questions/8368873/deserialize-json-string-generated-from-net-using-jackson}
%\url{http://stackoverflow.com/questions/10329706/json-deserialization-into-another-class-hierarchy-using-jackson}
%\url{http://stackoverflow.com/questions/14454028/polymorphic-serialization-of-collections-with-custom-serializer-in-jackson}
%\url{http://stackoverflow.com/questions/12350571/how-can-i-change-global-type-information-format-in-jackson}
%
%sprawdzić jak się wymienia ze wszystkim opisanym adnotacjami (dodać jeszcze rozszerzenie defaulttyperesolvera), wtedy sprawdzić mixed iny; lista objectów, na którą wpakuję stringi i inty

%\section{Restlet}
%\url{http://restlet.org/learn/tutorial/2.1/#/docs_2.0/13-restlet/275-restlet/266-restlet.html}\\
%\url{http://wiki.fasterxml.com/JacksonHowToIgnoreUnknown}\\

\section{Spring}
Javowy framework web serviceów.
%\section{Crest}
%\section{Ksoap}

%http://trac.e-technik.uni-rostock.de/projects/ws4d-gsoap/wiki/AndroidNDK

\section{I-jetty}
Serwer webowy dla Androida. Niby port Jetty. Może ma jakiś kontener Web-serviceów?

\section{Inne}
Chociaż wymienić pozostałe rozwiązania, które mogą być warte uwagi, choć raczej nie nadają się dla rozwiązania problemu tej pracy magisterskiej.

%\url{http://www.webopedia.com/TERM/W/Web_Services.html}\\
%\url{http://thrift.apache.org/static/files/thrift-20070401.pdf}\\
%reliable sessions (w sumie też by mi się przydał jakiś mechanizm, który umożliwi komunikację w niestabilnym środowisku Internetu)\\
%\url{http://blogs.msdn.com/b/shycohen/archive/2006/02/20/535717.aspx}\\
%\url{http://wsme.readthedocs.org/en/latest/}\\
%
%Apache Thrift, Google Protocol Buffers, ZeroMQ (\url{http://stackoverflow.com/questions/8062212/difference-between-apache-thrift-and-zeromq})\\
%\url{http://thrift.apache.org/static/files/thrift-20070401.pdf}\\
%\url{http://blogs.msdn.com/b/shycohen/archive/2006/02/20/535717.aspx}\\
%
%
%JiBX!!! Tworzenie klas ze schemy i robienie wsdl z javy\\
%Wsdl2Java: tworzenie klas z wsdl, głównie chodzi o klasy danych\\
%
%Serializacja:\\
%\url{http://simple.sourceforge.net/}\\
%\url{http://code.google.com/p/dbdroid-remoting/} - klient web serviców serializujący i deserializujący xmle\\
%Generacja javovego kodu z bindingu:\\
%XSD2Java – generuje kod java z xsd. Mocno niedorobione.\\
%Inne:\\
%\url{http://stackoverflow.com/questions/6920175/how-to-generate-java-classes-from-wsdl-file}\\
%\url{http://blog.tourgeek.com/2011/12/xml-data-binding-for-java-on-android.html}\\
%PODOBNE, ALE NIE PRZYDATNE:\\
%
%\url{http://forum.springsource.org/showthread.php?129058-Spring-Remoting-for-Android}\\
%\url{http://www.themidnightcoders.com/fileadmin/docs/java/v4/index.html?android}\\
%\url{http://developer.android.com/guide/components/aidl.html} - AIDL, androidowa komunikacja między procesami\\

%Python EVE (Rest framework)
%\url{http://www.blog.pythonlibrary.org/2014/08/13/jsonpickle-turning-python-pickles-into-json/}

\section{Porównanie możliwości tych technologii}
Zestawienie ogólne -- czy można coś zrobić z tym na Androidzie, czy możliwe są metody polimorficzne (i jak z resztą poszukiwanych przeze mnie cech ze wstępu), jaka jest moja subiektywna ocena pracy z danym frameworkiem, jak ciężko się instaluje, na ile potrafią współpracować z innymi rozwiązaniami (jak bardzo trzymają się standardów).

%Jakieś niestandardowe (third-party) narzędzia tworzące WSDLa z Javy i .NETa i z WSDLa klasy? To samo do schemy i ze schemy?
