\chapter{Eksperymenty (TODO)}
Zabawy z tym, co stworzyłem. Jakiś sensowny przykład użycia. Porównanie możliwości z innymi istniejącymi rozwiązaniami. Opis konfiguracji sprzętowej, na której testuję.

%Musi być lista drzew obiektów z różnymi normalnymi i ciężkimi przypadkami. Ta lista będzie używana do sprawdzania jak idzie serializacja, czy nie są tracone żadne informacje. Test polega na tym, że wysyłamy w jedną stronę z C\# (bo tam jest dobra informacja o typach) potem tamta strona to odczytuje, serializuje i~wysyła z~powrotem. Jak to, co wróciło jest takie samo, to test zaliczony.
%
%Udokumentować jak zestawiałem testy, podpinałem debuggery, zbierałem wyniki.
%Testy na Windowsie, Ubuntu i Androidzie
%
%Jak benchmarkować serwisy? Są jakieś kryteria, benchmarki? Trochę o błędności benchmarkowania
%
%Zrobić testy integracyjne w pythonie. Testy ze sleepami itd. maszyn zrobć przy pomocy automatyzacji Virtual Boxa.

%Mogę sprawdzić jak dodać bezpieczeństwo, albo niezawodność komunikacji (reliability).

Na jakim sprzęcie testuję.

\section{Przykład zastosowania}

\section{Porównanie funkcjonalności}
Jak moja biblioteka wypada na tle innych w realizacji celów, które sobie postawiłem? Też będzie tu bezpieczeństwo, wznawianie połączeń itp.

\section{Porównanie wydajności}
Radzenie sobie z dużym obciążeniem, szybkość w różnych warunkach. Zużycie zasobów.

\section{Porównanie nakładu pracy programisty}
Jak szybko można zrobić aplikację? Ile pracy wymaga połączenie dwóch aplikacji, jednej na Androidzie, drugiej w~.NETcie? Jak wypada to na tle konkurencji? Czy w ogóle konkurencja pozwala na coś takiego?