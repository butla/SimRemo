\documentclass[twoside,a4paper]{book}

\usepackage[pdftex]{graphicx}
\usepackage{amsmath}
\usepackage{amssymb}
\usepackage{textcomp}
\usepackage[utf8]{inputenc}
\usepackage[polish]{babel}
\usepackage[T1]{fontenc}
\usepackage{array}
% pakiet stosowany do url'i w bibliografii, zamienia odnośniki na ładnie sformatowane
\usepackage{url}
% pakiety służące do numerowania i tworzenia algorytmów
\usepackage{algorithmic}
\usepackage{algorithm}
% redefinicja etykiety nagłówkowej listy algorytmów, domyślna jest po angielsku
\renewcommand{\listalgorithmname}{Spis algorytmów}

% pakiet do wyliczania skali, przydatny przy dużych obrazkach
\usepackage{pgf}
% pakiet służący do automatycznego sortowania odnośników do bibliografii
\usepackage[sort]{natbib}
% tworzenie listingów
\usepackage{listings}
% tworzenie figur wewnątrz figur
\usepackage{subfig}
% do automatycznego skracania nazw rozdziałów i podrozdziałów używanych w nagłówkach strony by mieściły się w jednej linii
\usepackage[fit]{truncate}
% fancyhdr - ładne nagłówki, definicja wyglądu nagłówka, numery stron będą umieszczane w nagłówku po odpowiedniej stronie
\usepackage{fancyhdr}
\pagestyle{fancy}
\renewcommand{\chaptermark}[1]{\markboth{#1}{}}
\renewcommand{\sectionmark}[1]{\markright{\thesection\ #1}}
\fancyhf{}
\fancyhead[LE,RO]{\bfseries\thepage}
% tutaj ograniczamy szerokość pola w nagłówku zawierającego nazwę rozdziału/podrozdziału do 95% szerokości strony
% redefinicja sposobu prezentacji nazw domyślnie wypisywanych wielkimi literami (np. domyślnie w nagłówku Spis treści będzie miał postać SPIS TREŚCI)
% Uwaga! to może popsuć wielkie litery w ogóle! Jak coś nie działa należy usunąć \nouppercase{} z poniższych definicji
\fancyhead[LO]{\nouppercase{\bfseries{\truncate{.95\headwidth}{\rightmark}}}}
\fancyhead[RE]{\nouppercase{\bfseries{\truncate{.95\headwidth}{\leftmark}}}}
\renewcommand{\headrulewidth}{0.5pt}
\renewcommand{\footrulewidth}{0pt}

% definicja typu prostego wymagana przez pierwsze strony rozdziałów itp.
% powyższe reguły niestety tych stron nie dotyczą, gdyż Latex automatycznie przełącza je pomiędzy fancy a plain
% w tym wypadku eliminujemy nagłówki i stopki na stronach początkowych
\fancypagestyle{plain}{%
 \fancyhead{}
 \fancyfoot{}
 \renewcommand{\headrulewidth}{0pt}
 \renewcommand{\footrulewidth}{0pt}
}

\parskip 0.05in


% makro umożliwiające otaczanie symboli okręgami
\usepackage{tikz}
% brak justowania tekstu (bazą okręgu będzie linia tekstu)
\newcommand*\mycirc[1]{%
  \begin{tikzpicture}
    \node[draw,circle,inner sep=1pt] {#1};
  \end{tikzpicture}}

% pionowe justowanie tekstu, środek okręgu pokrywa się ze środkiem tekstu
\newcommand*\mycircalign[1]{%
  \begin{tikzpicture}[baseline=(C.base)]
    \node[draw,circle,inner sep=1pt](C) {#1};
  \end{tikzpicture}}

% zmiana nazwy twierdzeń i lematów
\newtheorem{theorem}{Twierdzenie}[section]
\newtheorem{lemma}[theorem]{Lemat}

% tworzenie definicji dowodu
\newenvironment{proof}[1][Dowód]{\begin{trivlist}
\item[\hskip \labelsep {\bfseries #1}]}{\end{trivlist}}
% \newenvironment{definition}[1][Definicja]{\begin{trivlist}
% \item[\hskip \labelsep {\bfseries #1}]}{\end{trivlist}}
% \newenvironment{example}[1][Przykład]{\begin{trivlist}
% \item[\hskip \labelsep {\bfseries #1}]}{\end{trivlist}}
% \newenvironment{remark}[1][Uwaga]{\begin{trivlist}
% \item[\hskip \labelsep {\bfseries #1}]}{\end{trivlist}}

% definicja czarnego prostokąta zwyczajowo dodawanego na koniec dowodu
\newcommand{\qed}{\nobreak \ifvmode \relax \else
      \ifdim\lastskip<1.5em \hskip-\lastskip
      \hskip1.5em plus0em minus0.5em \fi \nobreak
      \vrule height0.75em width0.5em depth0.25em\fi}

% poniższymi instrukcjami można sterować co ma być numerowane a co nie i co ma być wyświetlane w spisie treści
% \setcounter{secnumdepth}{3}
% \setcounter{tocdepth}{5}

% definicja czcionki mniejszej niż tiny (domyślnie takiej małej nie ma)
\usepackage{lmodern}
\makeatletter
  \newcommand\tinyv{\@setfontsize\tinyv{4pt}{6}}
\makeatother

% definicja jeszcze mniejszej czcionki
\usepackage{lmodern}
\makeatletter
  \newcommand\tinyvv{\@setfontsize\tinyvv{3.5pt}{6}}
\makeatother

% pakiet do obsługi wielostronicowych tabel
\usepackage{longtable}
\setlength{\LTcapwidth}{\textwidth}

\usepackage[section] {placeins}

\usepackage{multirow}

\usepackage{slantsc}

% nazwa pliku ze stroną tytułową
% \include{phd_titlepage}
% allows useg @ as a @ not as special character
% required for macro redefinition
\makeatletter

\usepackage{tabularx}

% parameters definition
% they cannot conflict with other
% like bibteh attributes etc.
%\def\promotor#1{\def\@promotor{#1}}
\def\miasto#1{\def\@miasto{#1}}
\def\studies#1{\def\@studies{#1}}
\def\descr#1{\def\@descr{#1}}
\def\indeks#1{\def\@indeks{#1}}
\def\dept#1{\def\@dept{#1}}
\def\speciality#1{\def\@speciality{#1}}

\def\maketitle{
  %removal of header
  \thispagestyle{empty}%

	% zmniejszenie marginesów, zeby strona byla wysrodkowana
	\addtolength{\hoffset}{-0.5cm}
	\addtolength{\voffset}{-1.5cm}
	\addtolength{\textwidth}{0.5cm}

  \begin{center}
    \begin{tabular}{lcr}
      \multirow{4}{*}{\includegraphics[height=2cm]{img/Politechnika-Gdanska-logo-2013.png}} & &
      \multirow{4}{*}{\includegraphics[height=2cm]{img/logo_eti.png}}\\
			& \textsc{\textbf{Politechnika Gdańska}} & \\    
      & \textsc{\textbf{Wydział Elektroniki, Telekomunikacji i Informatyki}}&\\
    \end{tabular}
  \end{center}
  \vspace{1.5cm}
	
	%\begin{large}
		\noindent
		\begin{tabular}{@{}lp{1cm}l@{}}
			\textbf{Katedra/Zakład:} & & \@dept\\
			\textbf{Kierunek studiów:} & & Informatyka\\
			\textbf{Specjalność:} & & \@speciality\\
			\textbf{Rodzaj studiów:} & & \@studies\\
			\textbf{Imię i nazwisko:} & & \@author\\
			\textbf{Nr albumu:} & & \@indeks\\
		\end{tabular}
	%\end{large}
	
  \begin{center}
    \vspace{1cm}
    \Large{\textbf{\uppercase{Praca dyplomowa magisterska}}}
  \end{center}
  \vspace{1cm}
	
	%\begin{large}
		\noindent\textbf{Temat pracy:} \\
		\@title\\
		\\
		\noindent\textbf{Zakres pracy:} \\
		\@descr\\
		
		\vspace{2.5cm}
		\noindent{}Potwierdzenie przyjęcia pracy:
		\vspace{1cm}
		
		\noindent
		\begin{tabular}{@{}lp{6cm}r@{}}
			Opiekun pracy: & & Kierownik Katedry/Zakładu: \\ 
			....................... & & ....................... \\
			....................... & & ....................... \\
			Tytuł, imię i nazwisko & & Tytuł, imię i nazwisko \\
		\end{tabular}
		
		\vspace*{\stretch{2}}
		\begin{center}
			\@miasto, \@date
		\end{center}
	%\end{large}
	
	%żeby strona z innym marginesem się zrobiła
	\pagebreak
	%wracamy z normalnym marginesem
	\addtolength{\hoffset}{0.5cm}
	\addtolength{\voffset}{1.5cm}
	\addtolength{\textwidth}{-0.5cm}
}

%restore @ sign
\makeatother

%\cleardoublepage

% parametry strony tytułowej, zdefiniowane są w plikach z poszczególnymi stronami
% tytuł pracy
\title{Zdalne wywoływanie metod języka Java w systemie Android z platformy .NET}
% autor
\author{Michał Bultrowicz}
% rok wydania
\date{2014}
% miasto, gdzie napisano pracę
\miasto{Gdańsk}
% promotor
%\promotor{dr inż.\ Jacek Lebiedź}
% wydział promotora, tylko dla phd_titlepage
% \promotordpt{Wydział Elektroniki, Telekomunikacji i~Informatyki}
% uczelnia promotora, tylko dla phd_titlepage
% \promotoruniv{Politechnika Gdańska}

% rodzaj studiów, tylko dla mgr_titlepage
\studies{Stacjonarne drugiego stopnia}
% opis pracy, tylko dla mgr_titlepage
\descr{Przegląd i~analiza istniejących rozwiązań problemu zdalnego wywoływania kodu w~heterogenicznych środowiskach. Stworzenie własnego obiektowego rozwiązania w~postaci bibliotek i~ewentualnych narzędzi pobocznych. Podstawowy model użycia obejmuje wywoływanie kodu na~Androidzie przez~program działający w~.NET. Nacisk jest kładziony na łatwość użycia powstałego oprogramowania w~nowych i~istniejących projektach.}
% nr indeksu, tylko dla mgr_titlepage
\indeks{119290}
% katedra, tylko dla mgr_titlepage
\dept{Inteligentnych Systemów Interaktywnych}
% specjalność
\speciality{Inteligentne Systemy Interaktywne}

% korekta marginesów - domyślnie latex ma jakieś kosmiczne
\usepackage{anysize}
\marginsize{3.5cm}{2.5cm}{2.5cm}{2.5cm}
% po zmianie marginesów konieczne jest wymuszenie przeliczenia nagłówków
\fancyhfoffset[E,O]{0pt}

\begin{document}
% sekcja wstępna książki, numerowana rzymskimi
\frontmatter
% generacja strony tytułowej załączonej wcześniej
\maketitle

% spis treści
\tableofcontents

% właściwa część książki, numerowana arabskimi od 1
\mainmatter

\chapter{Wstęp testowy}
\section{Idea dokumentu}

Dokument ten ma w~założeniu pomóc w pisaniu pracy magisterskiej/rozprawy doktorskiej w narzędziu jakim jest Latex. Zdecydowanie polecam jego używanie, gdyż znacznie upraszcza formatowanie i~skład tekstu, wykonując bardzo dużo pracy za autora. W dalszej części tego dokumentu na przykładach przedstawię kilka technik jak zrealizować rzeczy, na jakie natknąłem się w~trakcie pisania mojej pracy magisterskiej oraz rozprawy doktorskiej.

Dokument ten nie ma na celu dostarczenie kompletnej dokumentacji Latex'a, czy też zaznajomić z nim osoby nie mającego żadnej wiedzy na jego temat. Nie jest to samouczek (takich jest cała masa w sieci, np.~\cite{wikibooks}, polecam też zapoznanie się z dokumentacją pakietów, jakie zostały użyte w tym dokumencie), ale raczej zbiór ciekawych konstrukcji, jakie znalazłem w~trakcie pisania swojej pracy magisterskiej i rozprawy doktorskiej. Jednakże tam gdzie uznam to za stosowne będę podawał rzeczy również oczywiste, by osoby nie mające wcześniej kontaktu z Latex'em mogły, na bazie tego opracowania, rozpocząć pracę nad swoim dokumentem. Domyślnie przygotowany jest dla dwustronnego wydruku w stylu książki. Wydruk jednostronny wymaga zmiany definicji klasy z \textbf{twoside} na \textbf{oneside} i poprawienia definicji nagłówków i stopek (linie 35 -- 56 preambuły dokumentu). Polecam również zaglądać do kodu dokumentu. Sam dokument bardziej pokazywać będzie efekt końcowy, jaki można uzyskać. Samą treść najłatwiej podejrzeć już w pliku źródłowym.

Do pisania samego tekstu użyć można dowolnego edytora tekstu, pozwalającego na tworzenie zwykłych plików tekstowych. Polecam jednak jakieś zintegrowane środowisko umożliwiające szybką kompilację i~podgląd wynikowego dokumentu. Osobiście używam narzędzia o nazwie \textbf{Kile} dostępnego praktycznie w każdej dystrybucji systemu Linux.

Zalecam przyjrzenie się również nagłówkowi dokumentu -- zawiera on importy różnych przydatnych pakietów dodatkowych wraz z~opisem. Większość z nich będzie opisana lub wykorzystana w tym dokumencie.

Gorąco zachęcam do korzystania z Latex'a. Przy odrobinie wprawy jest to narzędzie wysoce wygodne i~wydajne, umożliwiające zdecydowanie prostsze tworzenie złożonych dokumentów niż dowolny WYSYWIG\@.

\section{Strony tytułowe}

Do dokumentu dołączyłem strony tytułowe podobne do oficjalnych formatek pracy magisterskiej (mgr\_titlepage.tex) oraz rozprawy doktorskiej(phd\_titlepage.tex). Pierwsza z nich, będąca bazą dla drugiej, opracowana została przez mgra inż.\ Michała Wójcika\footnote{\url{http://mwojcik.eti.pg.gda.pl}}.

\section{Poprawność dokumentu}

Przed drukiem warto zawsze sprawdzić swój dokument pod względem formalnej poprawności. Służy do tego narzędzie \textbf{lacheck}. Na stronach Katedry Architektury Systemów Komputerowych znajduje się interfejs WWW do tego narzędzia napisany przez mgra inż.\ Rafała Knopę. Narzędzie to jest dostępne pod adresem \url{http://kask.eti.pg.gda.pl/lacheck/}.

\section{Uwagi i poprawki}

Zachęcam wszystkich do modyfikacji, poprawiania i rozbudowy niniejszego dokumentu. Z Waszą pomocą z biegiem czasu dokument ten ma szansę stać się kompletnym przewodnikiem do Latexa pomocnym w trakcie pisania pracy magisterskiej czy doktorskiej. Wszelkie zmiany przesłane do mnie opublikuję na stronie katedralnej, oczywiście z zaznaczeniem współautorów.

\chapter{Wprowadzenie}
Tu będzie wstęp i teoria.

Do każdego punktu co i dlaczego znaleźć jakiś artykuł

\section{Wstęp}
Tutaj napiszę taki swój wstęp. Co chcę zrobić, jakie są wymagania, jakie mam pomysły na to.
Praca jest odpowiedzią na prawdziwy problem napotkany w pracy (opis, że mamy framework, pluginy, Android z innymi urządzeniami).
Jest trudniejsza niż inne prace, bo jest bardziej niezależna i dziedzina problemu jest szersza. Rzucony na otwartą wodę musiałem sobie radzić i tak może powstanie coś, czego jeszcze na świecie nie ma (albo nie jest tak przyjemne w użyciu).

Będę robił system oparty o JSON z oznaczaniem typów. Referencyjna implementacja w Pythonie, poza tym Java (Androidowa) i C\#. Każda wersja językowa zawiera zarówno klienta jak i serwer.

\section{Cel pracy}
Dokładnie określić po co ta praca. W czym ma się przydać, po co ją robić? Jakie ma wymagania?
Powiedzieć, że tematyka będzie wyjaśniona za jakiś czas.

CELE: pełne wsparcie programowania obiektowego, prostota i szybkość użycia, łączenie różnych technologii
Głównym celem pracy jest stworzenie rozwiązania (biblioteki), które pozwoli na zdalne wywoływanie kodu w systemie Android z platformy .NET przy pomocy języka C\#. Można powiedzieć, że jest to zarządzanie Androidem z Windowsa (bo .NET najprawdopodobniej chodzi właśnie na nim).

Druga rzecz, którą chcę uzyskać to możliwość łatwego tworzenia programów współpracujących między wyżej wymienionymi platformami. Kiedy wywołujemy zdalny kod często trzeba przekazać mu jakieś parametry, często też chcemy otrzymać wartości zwrotne. Pomijając sam fakt transportu danych z jednej platformy na drugą stajemy przed problemem niezgodności typów danych. Wiemy, że na obu końcach będą użyte dwa różne języki, więc struktury danych nie będą kompatybilne. Trzeba temu zaradzić.
Ogólnie ujęte rozwiązanie dla powyższych problemów to serwer usług internetowych (ang. web services) wraz z narzędziami. Dzięki mechanizmowi usług sieciowych można wykonać dowolny kod na serwerze nie zależnie od technologii wykorzystywanych po obu stronach. Nie istnieje jeszcze (przynajmniej nie udało się znaleźć) serwer web service’ów dla Androida. Istnieją za to biblioteki pozwalające na przetłumaczenie klas z C\# na Javę.

W reszcie rozdziału jest wytłumaczone dokładniej jak działają wspomniane tutaj mechanizmy oraz jak zostanie stworzony mój serwer.

Wiele środowisk chcemy wspierać, np. Windows Phone też

\section{Docelowe platformy}
Tu o tych, co będą wymaganie

\subsection{Android}
Tu o Androidzie, o JVM o samym języku Java. Jego środowisko programistyczne.

Wizja systemu Android. Że to na telefony. Co w tej Javie jest, dlaczego tak jest, czego nie ma (właśnie tych bibliotek). Dalvik. Co jest w C (bionic)?
Jak działa Java? Maszyna wirtualna, baza w C, teoretycznie ten sam kod powinien działać wszędzie, należy jednak dobrze zaimplementować podstawę wszystkich bibliotek (czyli maszynę). A czasem są np. rozbieżności (drobne) pomiędzy windowsem a Linuxem.

\subsection{.NET}
Tu o .NETcie i o C\# (bo tego będę używał)


\section{Zdalne wywoływanie kodu}
Tu o RPC, o problemach, o tłumaczeniu danych, istniejących standardach itp.

\subsection{Tłumaczenie danych}
Jaka jest problematyka przetłumaczenia danych/obiektów z jednego języka, platformy na drugą?
Nawet na niskim poziomie mamy little/big endian. Potem dochodzi jeszcze niezgodność kodów bajtowych platform wirtualnych.
Nawet jak się zapisze w XML, JSONie czy czymś innym to możemy stosować inne formaty, czy coś. Bindingi.

\emph{JAKAŚ KSIĄŻKA O TYM?}

\subsubsection{Serializacja}

\subsection{Transport danych}
Też bardzo szeroka dziedzina. Jak już się przetłumaczy, to zawsze trzeba jakoś przetransportować. Jak możemy przekazywać w ramach jednego kompa, jak pomiędzy kompami lokalnie, jak globalnie.
Celuję w globalnie, ale zwsze szerszą metodę można zastosować wężej (chociaż może nie być tak wydajna jak te węższe)

\emph{TEŻ PRZYDAŁBY SIĘ JAKIŚ ARTYKUŁ CHOCIAŻ}

\emph{TCP, PIPEy, SSL, HTTP, HTTPS, Webservicey, SOAP, REST, JSON}

\subsection{RPC}
Ogólnie trochę o RPC. Że wiele technologii to RPC (Web service'y, CORBA), że to zawiera i tłumaczenie i transport.
\url{http://www.infoq.com/articles/rest-soap-when-to-use-each}\\

\url{http://xmlrpc.scripting.com/default.html XML RPC}\\
\url{http://xmlrpc.scripting.com/spec.html}\\
\url{http://effbot.org/zone/xmlrpc-errata.htm}\\
Przejrzeć specyfikację, zobaczyć, z czym musieli sobie poradzić i wyczaić jak to wszystko mozna odwzorować w JSONie. Obczaić, jak sobie DataContractSerializer (też DataContractJsonSerializer) radzą. 

\subsection{XML binding}
NA TYM BĘDZIE DZIAŁAŁO TŁUMACZENIE KLAS
W WSDL poza opisami operacji realizowanych przez usługę znajdują się też pliki XSD opisujące wszystkie struktury danych (klasy), które będą wymieniane. Dla tego posiadając WSDL usługi możemy wygenerować wszystkie klasy potrzebne do stworzenia klienta usługi.

Złożony problem pojawia się wtedy, kiedy servicy mają działać w obie strony wykorzystując te same klasy danych. W bardziej skomplikowanych systemach rozproszonych może się pojawiać taka sytuacja. Załóżmy istnienie dwóch aplikacji: A i B. Są one napisane w różnych językach. Każda z nich udostępnia usługę zwracającą obiekt danych X. A korzysta z usługi wystawianej przez B, a B korzysta z usługi wystawianej przez A. Mimo tego, że obiekt X w obu aplikacjach będzie zawierał te same dane, to jednak jego kod będzie inny. Dlatego nie można mówić o jednej klasie X, a o dwóch. Nazwijmy je XA i XB. TUTAJ DALEJ WYTŁUMACZENIE PRZYKŁADU AŻ W KOŃCU DOJDZIEMY DO TEGO, ŻE TRZEBA NAPISAĆ KLASĘ W JEDNYM JĘZYKU, POTEM ZROBIĆ JEJ BINDING DO XSD I Z XSD DO DRUGIEGO JĘZYKA. 

\subsection{Web service'y i SOAP}
W sumie to przykład tłumaczenia danych  i transpotru.

Jeden ze sposobów na zdalne wywoływanie kodu to usługi internetowe. Korzystając z otwartych standardów takich jak XML, SOAP, WSDL i UDDI umożliwiają integrację aplikacji poprzez Internet. XML jest używany do etykietowania danych, SOAP do ich pakowania i transferu (alternatywnie można używać też RESTa lub JSONa), WSDL opisuje interfejs usługi, a UDDI dostarcza spisu dostępnych usług (on nie leży w naszym obszarze zainteresowań). SOAP i WSDL są zapisywane przy pomocy XML. Web servicy używane są najczęściej do komunikacji aplikacji komercyjnych (np. serwery, portale) ze sobą lub z klientami. Pozwalają organizacjom na wymianę danych nie wymagając wiedzy o wewnętrznej strukturze informatycznej za firewallem.

W przeciwieństwie do tradycyjnych modeli przetwarzania typu klient – serwer, takich jak np. system stron internetowych, usługi internetowe nie mają graficznego interfejsu użytkownika. Służą one do dzielenia logiki biznesowej, danych i procesów poprzez sieć, przy pomocy jednolitego interfejsu programistycznego. To aplikacje się komunikują, nie użytkownicy. Programiści mogą, co prawda, dodać usługę internetową do aplikacji, która posiada GUI (np. aplikacja okienkowa lub strona internetowa) aby zaoferować zwykłym użytkownikom jej funkcjonalność.

Usługi internetowe pozwalają różnym aplikacjom wykonanych w różnych technologiach (np. całkowicie odmienne języki programowania) na porozumiewanie się ze sobą bez potrzeby czasochłonnego tworzenia kodu tłumaczącego. Nie ma też znaczenia to, czy obie aplikacje znajdują się na jednej maszynie, czy po drugiej stronie świata.
Usługi sieciowe na platformie .NET realizowane są w ramach WCF. Javie obecnym standardem jest JAX-WS.

co to webservicy i po co są?\\
do czego mogą się przydać na androidzie?\\
dlaczego ich nie ma? Chyba stwierdzili, że ogólne web servicey są za ciężkie i za bardzo obciążają (Windows phone też nie ma WCF)\\

\subsection{REST}
Cośtam o RESTcie \url{http://www.oracle.com/technetwork/articles/javase/index-137171.html}\\
REST and POX \url{http://msdn.microsoft.com/en-us/library/vstudio/aa395208(v=vs.90).aspx}\\

Restful web services vs ``big'' web services: \url{http://www2008.org/papers/pdf/p805-pautassoA.pdf}\\

\chapter{Poszukiwanie rozwiązania}
Jak w ogóle dojść do tego  co i jak zrobić, żeby dało nam to, co mamy w celach?

Jak teoretycznie możnaby zrobić web service'y na Androidzie? Android jest tutaj przykładem ograniczonego systemu.
W sumie jakiś powód musięli mieć, żeby ich nie zamieszać. Jakie jeszcze dodatkowe usprawnienia byłyby do zrobienia (tłumaczenie klas)?

Najpierw zobaczmy, co jest.

\section{Podobne rozwiązania}
Tu o bibliotekach, frameworkach robiących to co chcemy albo rzeczy podobne.

Konkretne porównania osiągów w rozdziale o eksperymentach (chociaż tutaj już zrobić porównanie wysiłku programisty, po prostu zamieścić je na końcu). Tutaj napisać o funkcjonalności żeby w ogóle wiedzieć, czy to czegoś warte i na ile mam się inspirować. Poza tym, może się okazać że jest już coś, co robi to, ca ja bym chciał zrobić.

Te wszystkie do serializacji i soapów rzeczy to są, tutaj nic nie ma. Wszystko co by się chciało to jest, ale na dużą Javę, albo na pythona (ciężko przeportować, nie wiadomo, czy da się na pewno) Apache Thrift, Google Protocol Buffers, ZeroMQ (http://stackoverflow.com/questions/8062212/difference-between-apache-thrift-and-zeromq)

Przy każdej technologii pokazać implementacje w WCF i Javie normalnej (bo ona będzie takim punktem odniesienia) i zrobić porównania prędkości, bezpieczeństwa itp.

\url{http://www.webopedia.com/TERM/W/Web_Services.html}\\
\url{http://thrift.apache.org/static/files/thrift-20070401.pdf}\\
reliable sessions (w sumie też by mi się przydał jakiś mechanizm, który umożliwi komunikację w niestabilnym środowisku Internetu)\\
\url{http://blogs.msdn.com/b/shycohen/archive/2006/02/20/535717.aspx}\\
\url{http://docs.xamarin.com/guides/cross-platform/application_fundamentals/web_services}\\

Apache Thrift, Google Protocol Buffers, ZeroMQ (\url{http://stackoverflow.com/questions/8062212/difference-between-apache-thrift-and-zeromq})\\
\url{http://thrift.apache.org/static/files/thrift-20070401.pdf}\\
\url{http://blogs.msdn.com/b/shycohen/archive/2006/02/20/535717.aspx}\\
Xamarin\\
\url{http://docs.xamarin.com/guides/cross-platform/application_fundamentals/web_services/}\\

JiBX!!! Tworzenie klas ze schemy i robienie wsdl z javy\\
Wsdl2Java: tworzenie klas z wsdl, głównie chodzi o klasy danych\\

Serializacja:\\
\url{http://simple.sourceforge.net/}\\
\url{http://code.google.com/p/dbdroid-remoting/} - klient web serviców serializujący i deserializujący xmle\\
Generacja javovego kodu z bindingu:\\
XSD2Java – generuje kod java z xsd. Mocno niedorobione.\\
Inne:\\
\url{http://stackoverflow.com/questions/6920175/how-to-generate-java-classes-from-wsdl-file}\\
\url{http://blog.tourgeek.com/2011/12/xml-data-binding-for-java-on-android.html}\\
PODOBNE, ALE NIE PRZYDATNE:\\

\url{http://forum.springsource.org/showthread.php?129058-Spring-Remoting-for-Android}\\
\url{http://www.themidnightcoders.com/fileadmin/docs/java/v4/index.html?android}\\
\url{http://developer.android.com/guide/components/aidl.html} - AIDL, androidowa komunikacja między procesami\\


\subsection{WCF}
\subsection{JAX-WS}
Niestandardowe podejście z użyciem JSONa zamiast XMLa\url{http://jax-ws-commons.java.net/json/}\\

\subsection{Jackson}
\url{http://www.cowtowncoder.com/blog/archives/2010/03/entry_372.html}\\

\subsection{Restlet}
\url{http://restlet.org/learn/tutorial/2.1/#/docs_2.0/13-restlet/275-restlet/266-restlet.html}\\
\url{http://wiki.fasterxml.com/JacksonHowToIgnoreUnknown}\\

\subsection{Spring}
\subsection{Crest}
\subsection{Ksoap}

\subsection{I-jetty}
Jest to serwer, może ma jakiś kontener Web-serviceów?

\subsection{Porównanie możliwości tych technologii}


\section{Moje pomysły na rozwiązanie}
Wymienić, zrobić POCe, porównać, wybrać i uzasadnić wybór. Wypracować jaką metodę odcięcia pracy na POCem? Że jak będę widział, że da się coś zrobić, ale jest to za trudne to kończyć pracę.

Napisanie wszystkiego w C, Javie albo nawet Pythonie. Z czym by się to wiązało. Chcemy żeby było łatwo, co by trzeba było zrobić wtedy?
Podłączanie się pod istniejące rozwiązania Axis w C, portowanie Pythona, tłumaczenie JibXem

Zdecydowałem się to zrobić web servicami opartymi o SOAP, blabla, jednolite dla javy tej i normalnej
Tu już konkretnie jakie biblioteki, jak konkretniej to będzie wyglądać

a)	Rozpoznanie istniejących rozwiązań dotyczących wiązania danych (ang. data binding) z XML oraz wystawiania usług sieciowych (ang. web services) w językach Java oraz C++.
Zaprojektowanie i wykonanie biblioteki pozwalającej na przyjmowanie, przetwarzanie, tworzenie i wysyłanie wiadomości SOAP, generowanie plików WSDL na bazie kodu oraz na wiązanie danych z języka C\# do języka Java z zakresem bibliotek dostępnym na Androidzie. Akceptowalne jest również stworzenie biblioteki, w której komunikacja oparta zostałaby o protokół TCP/IP i serializację obiektów do XML tak, by były rozumiane zarówno przez kod w języku Java, jak i w języku C\#.

Jak jest? \\
JAVA dla przykładu \\
Co trzeba robić w javie			Co trzeba robić w C\# \\
ANDROID \\
Co trzeba zrobić na androidzie		Co trzeba robić w C\# \\

Jak może być? \\
Co robimy, żeby tak było? \\
Co trzeba wtedy robić na androidzie	Co w C\# \\

Może zamiast tylu podsekcji najpierw wyliczenie, krótki opis i decyzja, czy należy się tym zajmować? Dopiero te, które wydawałyby się bardziej interesujące dostają swoją własną podsekcję.

\subsection{własna metoda oparta o JSON z oznaczaniem typów}
Póki co wiem, że będzie JSON z opisanymi typami. Kanał komunikacyjny powinien być dowolny. 

\subsection{serwer web serviców z narzędziami do generacji kodu itp}
Generator wsdl na Androida + tworzenie bindingów do xsd + Parsowanie SOAPów

JiBX!!! Tworzenie klas ze schemy i robienie wsdl z javy
Wsdl2Java: tworzenie klas z wsdl, głównie chodzi o klasy danych

Weryfikacja: tworzę graf w xmlu. Są obiekty, które mogą mieć dzieci (tablice, listy). Próbuję wygenerować C\# dla porównania. Zapisuję grafy, które nie działają (w trakcie automatycznych testów). Zrobić wizualizator grafów.

Co by trzeba było po kolei zrobić w normalnej javie? (tłumaczenia do xsd, ujednolicanie namespaców, mamy do czynienia z jakimiś ArrayOfString, dziedziczenie chyba słabo działa przy generacji, nie da się używać DataContractSerializera żeby nie serializować części klas ogólnie syf, ale wszystko udokumentować). 

\subsection{Własny serwer REST}
Tu dyskusja o implementacji serwera RESTowego na Androidzie \url{https://groups.google.com/forum/?fromgroups=#!topic/android-developers/vgkXg1P8iBg}

\subsection{remoting przez serializację}
Po prostu serializujemy tak samo/

\subsection{portowanie rozwiązań w C na Androida?}
Axis. Może częściowe używanie go spod javy. W sumie można sportować też jakieś inne rzeczy. Np.\ pythonowe web service'y (pyro, twisted).

sl4a, py4a, kompilacja na ubuntu z użyciem NDK i Cythona
distutils.core zamienić na setuptools w setup.py

\subsection{Zcentralizowane pseudo web servicy (albo nawet i nie)}
Też potrzebny serializator / deserializator + bindingi. To był ten pomysł, że to komp pinga serwer (Androida), żeby ten sobie pobrał rozkaz i to Android woła web service na kompie. Trzeba by było używać Ksoap albo innego ścierwa. Zobaczyć, czy to w ogóle miałoby sens. Czy tymi Ksoapami da się coś sensownie zrobić.

\subsection{Wybrane rozwiązanie}
Jakie i dlaczego?

Dlaczego nie soapem?

\chapter{Projekt systemu}
Tutaj cały czas teorie, ale już wiadomo na czym się skupić. Wiem, jaki chcę zrobić system, teraz muszę go zaprojektować.

Jakiej licencji użyję?

Call:
- id obiektu (0 dla tworzenia nowego, wtedy argumentem jest tym)
- id metody
- parametry

Response:
- czy sie udalo
- obiekt zwrotny, opis błędu przy wywołaniu zdalnej metody lub stacktrace z exceptiona, który wystąpił w metodzie

Id metody:
składa się z nazwy i aliasów jej parametrów.

Alias:
typy prymitywne - int, string, itp.
datacontract - sraka:\#ptaka.daka
lista - List
tablica - [int, [[int, [sraka:\#ptaka.daka, itp.

Tłumaczenie aliasów:
C\# tłumaczy klasy na aliasy, jak przetłumaczy dla jednej metody to dodaje to do słownika, żeby potem robić to od razu.
Java tłumaczy aliasy na nazwy klas (potem loaderem). Wiąże w słowniku id metody z obiektem metody. Może być metoda przyjmująca jakiś konkretny List, np. ArrayList, ale nie może mieć overloada przyjmującego inny List.
Kiedy czyścić te słowniki?

\section{Specyfikacja wymagań systemowych}
Wymienić te wszystkie typy wymagań i wylistować.

reliable sessions (w sumie też by mi się przydał jakiś mechanizm, który umożliwi komunikację w niestabilnym środowisku Internetu)

\section{Przypadki użycia}

\section{Struktura projektu}
Diagramy warstwowe, komponentowe, klas.

\section{Schemat działanie projektu}
Schematy działania, to z liniami życia itp.


\chapter{Implementacja}
Jak mi wyszło, jak wrażenia? Co musiałem zmienić, co dopracować, co dookreślić? Są jakieś poprawki do projektu? (go samego nie zmieniać).

\section{Narzędzia}
Jakie repo, visuale, notepad++ itp.

\chapter{Eksperymenty}

\section{Porównanie funkcjonalności}
Będzie tu też bezpieczeństwo, wznawianie połączeń, radzenie sobie z obiążeniem. Takie główne funkcjonalności już były omawiane wcześniej.

\section{Porównanie wydajności}
Radzenie sobie z dużym obciążeniem, szybkość w różnych warunkach. Zużycie zasobów.

\section{Porównanie nakładu pracy programisty}


\chapter{Podsumowanie}
I co? Udało się zrobić system który robi to, co miał robić? Zrobi furorę na świecie i będę bogaty?

\backmatter

% lista rysunków
\listoffigures
% lista tabel
\listoftables

% rodzaj bibliografii
\bibliographystyle{plain}
% plik z wpisami bibliograficznymi
\bibliography{bibliografia}
\end{document}