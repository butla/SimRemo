\chapter{Implementacja (TODO)}

\section{Licencja}
Ważna rzecz w~dzisiejszym oprogramowaniu (źródło).

Wybrałem MIT license (sam przeczytałem licencje, a porównanie tutaj http://choosealicense.com/licenses/). Jest przeciwnikiem skomplikowanych przepisów, na które Apache (główny konkurent) zakrawa, a na patentach mi nie zależy (wątpię, żebym jakieś tu miał, poza tym w UE nie można patentować kodu, chyba). MIT jest proste, zapewnia wzmiankowanie mojego nazwiska i umożliwia używanie mojego kodu także w komercyjnych rozwiązaniach.

Używa jej Ruby On Rails i parę innych dużych, znanych projektów.


\section{Narzędzia}
Narzędzia, których używałem/użyję. Jakie repozytorium, IDE, kompilatory, Fiddler itp.

\subsection{Kontrola wersji}
Musiałem to wybrać przed rozpoczęciem pisania tej pracy, bo chciałem ją tam przechowywać (backup). Sposób przechowywania źródeł to też ważna rzecz (link).
Nawet w przypadku samego dokumenty pomaga mi śledzić zmiany i rozwój oraz zapewnia kopię zapasową (a więc bezpieczeństwo).
Założyłem repozytorium na GitHubie, którego znam. Jest popularny i wygodny w użyciu. Kod ma być publiczny, więc nie mam problemu.
Git integruje się z~większością IDE.


\subsection{Maszyna deweloperska}
Opisać na jakim komputerze, z~jakim OSem to wszystko pisałem i~testowałem.
Ewentualne fizyczne Androidy też trzeba przedstawić


\subsection{IDE}
Eclipse z ADT
Visual Studio 2013


\subsection{Emulator Androida}
\label{android-emulator}
Używałem Eclipsa z wtyczką do Androida z~\url{http://developer.android.com/sdk/index.html}.
Ściągnąłem i używam narzędzia dla Androida 4.4.2.

Emulator chodził na obrazie do tych narzędzi, czyli \emph{ARM EABI v7a}. Emulator standardowy od Google.\footnote{Jest jeszcze alternatywny emulator wzmiankowany na stronach Xamarina -- Genymotion, \url{http://www.genymotion.com/}. Kto wie, może są jeszcze inne. Można też chyba w~VirtualBoxie hostować Androida na x86}
Używam ARMowego, bo więcej telefonów jest właśnie na nim. Zdażają się też drobne rozbieżności działania niektórych niskopoziomowych aplikacji względem obrazu na architekturę x86 (\emph{Intel x86 Atom}).
Konfiguracja używanej AVD: (w obrazku)
Forward portu



\section{Przebieg prac}
Jak mi wyszło, jak wrażenia? Co musiałem zmienić, co dopracować, co dookreślić? Są jakieś poprawki do projektu?

\subsection{Metodyka}
Robiłem Test Driven Development. Daje zazwyczaj dobre programy i~skutkuje w~końcu szybszym wykonaniem projektu, bo nie ma przerw w~pracy związanych z jakimiś dziwnymi bugami.
Cała podsekcja na to potrzebna?

\subsection{Szczegóły implementacji}
Co ciekawsze zabiegi, które zastosowałem.
Jakie logowanie?

\subsection{Wykonywanie testów}
Jak robiłem te zaplanowane testy?

%Przy funkcjonalnych testach jako kanał komunikacji może służyć strumień w pamięci. W unittestach oczywiście kanały zmockowane.