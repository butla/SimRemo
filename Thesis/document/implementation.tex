\chapter{Implementacja (TODO)}

\section{Narzędzia}
Narzędzia, których używałem/użyję. Jakie repozytorium, IDE, kompilatory, Fiddler itp.

Eclipse z ADT
Komputer mój opisać, jako maszynę deweloperską i~testową.

\subsection{Emulator Androida}
\label{android-emulator}
Używałem Eclipsa z wtyczką do Androida z~\url{http://developer.android.com/sdk/index.html}.
Ściągnąłem i używam narzędzia dla Androida 4.4.2.

Emulator chodził na obrazie do tych narzędzi, czyli \emph{ARM EABI v7a}. Emulator standardowy od Google.\footnote{Jest jeszcze alternatywny emulator wzmiankowany na stronach Xamarina -- Genymotion, \url{http://www.genymotion.com/}. Kto wie, może są jeszcze inne. Można też chyba w~VirtualBoxie hostować Androida na x86}
Używam ARMowego, bo więcej telefonów jest właśnie na nim. Zdażają się też drobne rozbieżności działania niektórych niskopoziomowych aplikacji względem obrazu na architekturę x86 (\emph{Intel x86 Atom}).
Konfiguracja używanej AVD: (w obrazku)
Forward portu

\section{Przebieg prac}
Jak mi wyszło, jak wrażenia? Co musiałem zmienić, co dopracować, co dookreślić? Są jakieś poprawki do projektu?

\subsection{Szczegóły implementacji}
Co ciekawsze zabiegi, które zastosowałem.

\subsection{Założenia przy testach}
Co tak naprawdę testuję? Jak, pod jakim kątem?

Przy funkcjonalnych testach jako kanał komunikacji może służyć strumień w pamięci. W unittestach oczywiście kanały zmockowane.